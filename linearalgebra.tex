\documentclass[10pt,landscape]{scrartcl}
\usepackage{multicol}
\usepackage{calc}
\usepackage{ifthen}
\usepackage[landscape]{geometry}
\usepackage{amsmath,amsthm,amsfonts,amssymb}
\usepackage{color,graphicx,overpic}
\usepackage{hyperref}


% This sets page margins to .5 inch if using letter paper, and to 1cm
% if using A4 paper. (This probably isn't strictly necessary.)
% If using another size paper, use default 1cm margins.
\ifthenelse{\lengthtest { \paperwidth = 11in}}
    { \geometry{top=.5in,left=.5in,right=.5in,bottom=.5in} }
    {\ifthenelse{ \lengthtest{ \paperwidth = 297mm}}
        {\geometry{top=1cm,left=1cm,right=1cm,bottom=1cm} }
        {\geometry{top=1cm,left=1cm,right=1cm,bottom=1cm} }
    }

% Turn off header and footer
\pagestyle{empty}

% Redefine section commands to use less space
\makeatletter
\renewcommand{\section}{\@startsection{section}{1}{0mm}%
                                {-1ex plus -.5ex minus -.2ex}%
                                {0.5ex plus .2ex}%x
                                {\normalfont\large\bfseries}}
\renewcommand{\subsection}{\@startsection{subsection}{2}{0mm}%
                                {-1explus -.5ex minus -.2ex}%
                                {0.5ex plus .2ex}%
                                {\normalfont\normalsize\bfseries}}
\renewcommand{\subsubsection}{\@startsection{subsubsection}{3}{0mm}%
                                {-1ex plus -.5ex minus -.2ex}%
                                {1ex plus .2ex}%
                                {\normalfont\small\bfseries}}
\makeatother

% Define BibTeX command
\def\BibTeX{{\rm B\kern-.05em{\sc i\kern-.025em b}\kern-.08em
    T\kern-.1667em\lower.7ex\hbox{E}\kern-.125emX}}

% Don't print section numbers
\setcounter{secnumdepth}{0}


\setlength{\parindent}{0pt}
\setlength{\parskip}{0pt plus 0.5ex}

%My Environments
\newtheorem{example}[section]{Example}
% -----------------------------------------------------------------------

\author{Gerold}
\begin{document}
\raggedright
\footnotesize
\begin{multicols}{3}


% multicol parameters
% These lengths are set only within the two main columns
%\setlength{\columnseprule}{0.25pt}
\setlength{\premulticols}{1pt}
\setlength{\postmulticols}{1pt}
\setlength{\multicolsep}{1pt}
\setlength{\columnsep}{2pt}

\begin{center}
     \Large{\underline{Linear Algebra cheat sheet}} \\
\end{center}

\section{Vectors}
dot product: $u * v = ||u|| * ||v|| * cos (\phi) = u_x v_x + u_y v_y$\\
cross product: $u \times v = \left (\begin{array}{c}u_y v_z - u_z v_y\\u_z v_x - u_x v_z\\u_x v_y - u_y v_x\\\end{array}\right)$

enclosed angle:
\begin{align*}
cos \phi = \frac{u * v}{||u|| * ||v||}\\
||u|| * ||v|| = \sqrt {(u^2_x + u^2_y) (v^2_x + v^2_y)}
\end{align*}


\section{Matrices}
\subsection{determinants}
\[det (A \cdot B) = det (A) \cdot det (B)\]\\
TODO: other rules

\subsection{common properties}
symmetric: $A = A^T$\\
orthogonal: $A^T = A^{-1}$\\
diagonal: Eigenvalues on main diagonale

\subsubsection{regular / invertable / nonsingular}
\[det(A)^{-1} = det(A^{-1})\]\\

\subsubsection{diagonalizable}
If A can be diagonalized:
\[
P^{-1}AP=\begin{pmatrix}\lambda_{1}\\
& \ddots\\
& & \lambda_{n}\end{pmatrix}
\]
then:
\[AP=P\begin{pmatrix}\lambda_{1}\\
& \ddots\\
& & \lambda_{n}\end{pmatrix}\]

\subsubsection{triangular}


\end{multicols}
\end{document}